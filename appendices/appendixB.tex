\chapter{Numerical toolbox for quantum interferometry}
\label{app:numsingle}

In this appendix we present a collection of algorithms that together allow to reproduce the numerical results for optical interferometry in chapters \ref{chap:nonasymptotic}, \ref{chap:limited} and \ref{chap:future}. Most of the codes are written in MATLAB, the only exception being the second algorithm in appendix \ref{sec:singleshotalgorithm}, which has been developed in Mathematica.

\section{Basic elements and initial states}
\label{sec:elementsnum}

Operators in the space of a single electromagnetic mode:
\begin{lstlisting}[language=Matlab]
function [creat] = creation(dimension)
% Matrix representation of the creation operator, where 'dimension' is
% the cutoff of the space.
creat=zeros(dimension,dimension);
for aa=1:dimension
    for bb=1:dimension
        if aa == (bb+1); creat(aa,bb)=sqrt(bb);
        else; creat(aa,bb)=0;
        end
    end
end
creat=sparse(creat); end
\end{lstlisting}

\begin{lstlisting}[language=Matlab]
function [id] = identity(dimension)
% Identity matrix
id=eye(dimension);
id=sparse(id); end
\end{lstlisting}

\justify{Optical elements:}
\begin{lstlisting}[language=Matlab, mathescape=true]
function [j1] = j1schwinger(dimension)
% Matrix representation of the J1 operator (Jordan-Schwinger map).
j1=0.5*(kron(creation(dimension),creation(dimension)') + kron(creation$\hspace{0.15em}\swarrow$
(dimension)',creation(dimension)));
j1=sparse(j1); end
\end{lstlisting}

\begin{lstlisting}[language=Matlab, mathescape=true]
function [j3] = j3schwinger(dimension)
% Matrix representation of the J3 operator (Jordan-Schwinger map).
j3=0.5*(kron(creation(dimension)*creation(dimension)',identity$\hspace{0.15em}\swarrow$
(dimension))-kron(identity(dimension),creation(dimension)*creation$\hspace{0.15em}\swarrow$
(dimension)'));
j3=sparse(j3); end
\end{lstlisting}

\begin{lstlisting}[language=Matlab]
function [v] = beam_splitter(dimension)
% Matrix representation of a 50:50 beam splitter. 
v=expm(-1i*0.5*pi*sparse(j1schwinger(dimension)));
v=sparse(v); end
\end{lstlisting}

\begin{lstlisting}[language=Matlab]
function [utheta] = phase_shift_diff(dimension, theta)
% Matrix representation of the unitary encoding of the unknown parameter 
% 'theta' (difference of phase shifts).
utheta=expm(-1i*theta*j3schwinger(dimension));
utheta=sparse(utheta); end
\end{lstlisting}

\justify{Initial probe states for a Mach-Zehnder interferometer:}
\begin{lstlisting}[language=Matlab]
function [zero] = vacuum(dimension)
% Vacuum state for a single mode.
temp=identity(dimension);
zero=temp(:,1);
zero=sparse(zero); end
\end{lstlisting}

\begin{lstlisting}[language=Matlab]
function [displ] = displacement(dimension,alpha)
% Matrix representation of the displacement operator, where 'alpha' is
% the amount of displacement. 
displ=expm(alpha*creation(dimension)-conj(alpha)*creation(dimension)');
displ=sparse(displ); end
\end{lstlisting}

\begin{lstlisting}[language=Matlab, mathescape=true]
function [squ] = squeeze(dimension,zeta)
% Matrix representation of the squeezing operator for a single mode, 
% where 'zeta' is the squeezing parameter.
squ=expm(0.5*(conj(zeta)*(creation(dimension)')^2-zeta*(creation$\hspace{0.15em}\swarrow$
(dimension))^2));
squ=sparse(squ); end
\end{lstlisting}

\begin{lstlisting}[language=Matlab, mathescape=true]
function  [initial_state] = initial_probe(state_sel)
% Common states in optical interferometry, where  'state_sel' is a number
% from 1 to 5 labelling the quantum probes
%
%   (1) Coherent state: |alpha/sqrt(2),-i*alpha/sqrt(2)>
%   (2) NOON state: (|N,0>+|0,N>)/sqrt(2)
%   (3) Twin squeezed vacuum state: S_a(z)S_b(z)|0,0>
%   (4) Squeezed entangled state: N (|z,0> + |0,z>)
%   (5) Twin squeezed cat state: [N S(z)(|alpha>+|-alpha>)]\otimes2
%
% whose componentes are generated in the number basis of a Mach-Zehnder 
% interferometer.
%
% The code is configured with the parameters
%
%   (1) alpha=sqrt(2)
%   (2) N=2
%   (3) z=asinh(1)
%   (4) z=log(2+sqrt(3))
%   (5) alpha=0.960149, z=1.2145
%
% so that the nbar number of quanta that enters the interferometer is 2.
%
% The cutoff for the vectors are: (1) 20, (2) 2, (3) 50, (4) 60 
% and (5) 50. These values are selected such that the numerical states 
% are a reasonable approximation to the analytical kets.

% State parameters
nbar=2;
number=nbar; % Mean number of photons
alpha=sqrt(nbar); % Displacement parameter
zeta=asinh(sqrt(nbar/2)); % Squeezing parameter
zent=log(2+sqrt(3));

alphacat=0.960149; % Maximum Fisher information for the twin squeezed 
zcat=1.2145;         % cat state

%alphacat=1.09048; % Same Fisher information for the twin squeezed cat 
%zcat=1.1025;        % state and the squeezed entangled state

if state_sel==1
    num_cutoff=20; % Cutoff for states
elseif state_sel==2
    num_cutoff=number;
elseif state_sel==3 || state_sel==5
    num_cutoff=50;
elseif state_sel==4
    num_cutoff=60;
end
op_cutoff=num_cutoff+1; % Cutoff for operators

% Initial state
if state_sel==1
    initial_temp=sparse(displacement(op_cutoff,alpha)*vacuum(op_cutoff));
    initial_state=sparse(kron(initial_temp,vacuum(op_cutoff)));
    initial_state=beam_splitter(op_cutoff)*initial_state;
elseif state_sel==2
    initial_temp=sparse((creation(op_cutoff)^number)*vacuum(op_cutoff));
    initial_state=sparse(kron(initial_temp,vacuum(op_cutoff))+kron(vacuum$\hspace{0.15em}\swarrow$
(op_cutoff),initial_temp));
elseif state_sel==3
    initial_temp1=sparse(squeeze(op_cutoff,zeta)*vacuum(op_cutoff));
    initial_temp2=sparse(squeeze(op_cutoff,zeta)*vacuum(op_cutoff));
    initial_state=sparse(kron(initial_temp1,initial_temp2));
elseif state_sel==4
    initial_state=(kron(squeeze(op_cutoff,zent),identity(op_cutoff))$\hspace{0.15em}\swarrow$         	
+kron(identity(op_cutoff),squeeze(op_cutoff,zent)))*kron(vacuum$\hspace{0.15em}\swarrow$
(op_cutoff),vacuum(op_cutoff));
elseif state_sel==5
    initial_state=squeeze(op_cutoff,zcat)*(displacement$\hspace{0.15em}\swarrow$
(op_cutoff,alphacat)+displacement(op_cutoff,-alphacat))*vacuum(op_cutoff);
    initial_state=kron(initial_state,initial_state);
    initial_state=initial_state/sqrt((initial_state'*initial_state));
end
initial_state=initial_state/sqrt((initial_state'*initial_state)); end
\end{lstlisting}

\section{Optimal quantum strategies for the square error criterion}
\label{sec:singleshotalgorithm}

The following algorithm has been utilised to calculate the optimal single-shot strategies in chapter \ref{chap:limited}:
\begin{enumerate}
\item The components $c_{nm}$ of $\ket{\psi_0}$ are numerically approximated in a finite space of dimension $d_c$ per mode. For the coherent state this dimension is $d_c=21$, and the number probability for this cutoff is $p_c \sim 10^{-19}$; for the twin squeezed vacuum state we have that $d_c = 51$ and $p_c \sim 10^{-17}$; $d_c = 61$ and $p_c \sim 10^{-5}$ for the squeezed entangled state; and $d_c = 51$ and $p_c \sim 10^{-10}$ for the twin squeezed cat state. This is achieved via the code for generating initial states in the previous appendix.
\item $\mathcal{K}$ and $\mathcal{L}$ are numerically generated using the formulas in equations (\ref{k_semianalytical_sol}) - (\ref{singularcases}). This allows us to calculate $\rho = \rho_0 \circ \mathcal{K}$ and $\rho = \rho_0 \circ \mathcal{L}$ in the number basis.
\item The basis of $\rho$ and $\bar{\rho}$ is changed as $\rho_D = \mathcal{V}^\dagger \rho \mathcal{V}$ and $\bar{\rho}_D = \mathcal{V}^\dagger \bar{\rho} \mathcal{V}$, where the columns of $\mathcal{V}$ are given by the eigenvectors $\ket{\phi_i}$ of $\rho$, $(\rho_D)_{ij} = p_i\delta_{ij}$ and $(\bar{\rho}_D)_{ij} = \bra{\phi_i} \bar{\rho} \ket{\phi_j}$.
Only the eigenvectors $\ket{\phi_i}$ whose eigenvalues $p_i$ satisfy that $p_i \gtrsim 10^{-12}$ are employed. 
\item Now we can calculate the elements $(S_D)_{ij} = \bra{\phi_i} S \ket{\phi_j}=2(\bar{\rho}_D)_{ij}/(p_i + p_j)$ directly.
\item We return to the original basis using $S=\mathcal{V} S_D \mathcal{V}^\dagger$.
\item Finally, we calculate the spectral decomposition of $S$, which gives us the estimates $\lbrace s\rbrace$ and the projectors $\lbrace \ket{s}\rbrace$.
\end{enumerate}
Its implementation in MATLAB is:

\begin{lstlisting}[language=Matlab, mathescape=true]
function [Sopt,sopt,soptvec_columns,bayes_bound,rho,pk,psik,rhobar,$\hspace{0.15em}\swarrow$
rhobarnew] = mz_optimal_1trial(initial_state,phase_width,phase_mean)
% Optimal single-shot strategy, where 'initial_state' is a pure state 
% for the Mach-Zehnder interferometer, 'phase_width' is the width of the 
% parameter domain and 'phase_mean' is its centre.
%
% This programme calculates:
%
%   a) the optimal quantum estimator 'Sopt'
%   b) the estimates 'sopt' for the unknown parameter given by the
%      spectrum of 'Sopt'
%   c) the optimal projective measurement for a single trial given by
%      the eigenvectors 'soptvec_columns' of 'Sopt'
%   d) the optimal single-shot mean square error 'bayes_bound'
%   e) the zero-th quantum moment of the transformed density matrix 'rho'
%   f) 'rho' in its diagonal basis, denoted by 'pk'
%   g) the matrix 'psik' whose columns are the eigenvectors of 'rho'
%   h) the first quantum moment of the transformed density matrix 'rhobar'
%   i) 'rhobar' in the eigenbasis of 'rho', denoted by 'rhobarnew'

% Calculation of 'rho' and 'rhobar'
index=1;
kvec=zeros(1,length(initial_state)^2);
lvec=zeros(1,length(initial_state)^2);
for x1=1:sqrt(length(initial_state))
    for y1=1:sqrt(length(initial_state))
        for z1=1:sqrt(length(initial_state))
            for t1=1:sqrt(length(initial_state))
                if (x1-1)-(y1-1)+(t1-1)-(z1-1)==0
                    K=phase_width;
                    L=phase_mean*phase_width;
                else
                    comp_temp=(x1-1)-(y1-1)+(t1-1)-(z1-1);
                    exp_temp=exp(-1i*comp_temp*phase_mean/2);
                    sin_temp=sin(comp_temp*phase_width/4);
                    cos_temp=cos(comp_temp*phase_width/4);
                    K=4*exp_temp*sin_temp/comp_temp;
                    L=exp_temp*(4*phase_mean*sin_temp/comp_temp$\hspace{0.15em}\swarrow$ 
+1i*2*phase_width*cos_temp/comp_temp - 1i*8*sin_temp/comp_temp^2);
                end
                kvec(index)=K/phase_width;
                lvec(index)=L/phase_width;
                index=index+1;
            end
        end
    end
end

kmat=sparse(vec2mat(kvec,sqrt(length(kvec))));
lmat=sparse(vec2mat(lvec,sqrt(length(lvec))));
initial_rho=kron(initial_state,initial_state');
rho=initial_rho.*kmat; rhobar=initial_rho.*lmat;
rho=full(rho); rhobar=full(rhobar);

% Eigenvalues and eigenvectors of 'rho'
[psik, pk] = eigs(rho,rank(rho));
psik=sparse(psik);
pk=sparse(pk);

pkvec=zeros(1,length(pk));
for x=1:length(pk)
    pkvec(x)=pk(x,x);
end

% 'rhobar' in the eigenbasis of 'rho'
rhobarnew=psik'*rhobar*psik;

% Optimal single-shot strategy: projectors and outcomes
Sopt_temp=zeros(length(pkvec),length(pkvec));
for a=1:length(pkvec)
    for b=1:length(pkvec)
        if pkvec(a)+ pkvec(b)>0
            Sopt_temp(a,b)=2*rhobarnew(a,b)/(pkvec(a)+pkvec(b));
        end
    end
end

Sopt_temp=sparse(Sopt_temp);
Sopt=psik*Sopt_temp*psik';
Sopt=full(Sopt);

[soptvec_columns, sopt_temp] = eigs(Sopt,rank(Sopt));
sopt=zeros(1,length(sopt_temp));
for x=1:length(sopt_temp)
    sopt(x)=sopt_temp(x,x);
end
soptvec_columns=sparse(soptvec_columns);
sopt=sparse(sopt);
if imag(sopt)<1e-5; sopt=real(sopt);
else
    error('The estimates of the unknown parameter must be real. Check$\hspace{0.15em}\swarrow$
the cutoff in the intermediate calculations.')
end

% Phase domain
phase=linspace(phase_mean-phase_width/2,phase_mean+phase_width/2,1000);

% Optimal single-shot mean square error
bayes_bound=trapz(phase,phase.*phase)/phase_width - trace(Sopt*Sopt*rho);
if imag(bayes_bound)<1e-10; bayes_bound=real(bayes_bound);
else
    error('The mean square error must be real. Check the cutoff$\hspace{0.15em}\swarrow$
in the intermediate calculations.')
end

end
\end{lstlisting}

The previous algorithm was extended in our work \cite{jesus2018} to calculate the collective measurement that is optimal on $\mu$ copies of a NOON state (see section \ref{subsec:optnoon}). However, a more economic alternative using Mathematica is:

\begin{mathematicathesis}
(* Optimal single-shot mean square error for collective POMs on 'mu'$\hspace{0.15em}\swarrow$
copies of a NOON state *)

(* Number of copies *)
Clear[mu]
mu = 4; (* 'mu' must be greater than or equal to 2 *)

(* Transformed density matrix *)
Clear[nbar, rhotemp, rhotheta]
nbar = 2;
rhotemp[theta_] := N[ {{1/2, Exp[-I*nbar*theta]/2},$\hspace{0.15em}\swarrow$
{Exp[I*nbar*theta]/2, 1/2}}]; 
rhotheta[theta] = KroneckerProduct[rhotemp[theta ], rhotemp[theta ]];
Do[rhotheta[theta ] = KroneckerProduct[rhotemp[theta ],$\hspace{0.15em}\swarrow$
rhotheta[theta ]], {j, mu - 2}]; (* 'j' is the index of repetition *)

(* Prior probability *)
Clear[priorwidth, priormean, a, b, prior]
priorwidth = Pi/2.;
priormean = 0;
a = priormean - priorwidth/2;
b = priormean + priorwidth/2;
prior[theta_] := 1/(b - a);

(* Calculation of 'rho' and its diagonal form*)
Clear[rho, rhodiag]
rho = Integrate[prior[theta]*rhotheta[theta], {theta, a, b}];
rhodiag = DiagonalMatrix[Eigenvalues[rho]];

(* Calculation of 'rhobar' and its form in the eigenbasis of rho*)
Clear[rhobar, change, rhobardiag]
rhobar = Integrate[prior[theta]*rhotheta[theta]*theta, {theta, a, b}];
change = Transpose[Eigenvectors[rho]];
rhobardiag = Inverse[change].rhobar.change;

(* Optimal quantum estimator *)
Clear[rhosupp, rhobarsupp, Sopt]
rhosupp = rhodiag[[1 ;; MatrixRank[rho], 1 ;; MatrixRank[rho]]];
rhobarsupp = rhobardiag[[1 ;; MatrixRank[rho], 1 ;; MatrixRank[rho]]];
Sopt = LyapunovSolve[rhosupp/2., rhosupp/2., rhobarsupp];

(* Optimal single-shot mean square error *)
Clear[emse]
emse = Round[NIntegrate[prior[theta]*theta^2, {theta, a, b}]$\hspace{0.15em}\swarrow$ 
- Tr[Sopt.Sopt.rhosupp], 10.^(-7)]

0.0428159 (* Result for mu = 4 *)
\end{mathematicathesis}

Furthermore, we notice that the single-shot calculation for a lossy interferometer in section \ref{loss} was carried out with a similar version of the Mathematica code above.

\section{Other quantum bounds}

\subsection{Quantum Cram\'{e}r-Rao bound}
\label{subsec:qcrbmatlab}

\begin{lstlisting}[language=Matlab]
function [qcrb] = mz_qcrb(initial_state,mu_max)
% Quantum Cramer-Rao bound as a function of the number of trials, where
% 'initial_state' is a pure state for the Mach-Zehnder interferometer
% and 'mu_max' is the maximum number of repetitions.

% Number of repetitions
observations=1:1:mu_max;

% Space cutoff (for a single mode)
op_cutoff=sqrt(length(initial_state));

% Quantum Fisher information (pure states and unitary encoding)
expectation_n=initial_state'*j3schwinger(op_cutoff)*initial_state;
expectation_n2=initial_state'*j3schwinger(op_cutoff)^2*initial_state;
qfi=4*(expectation_n2-expectation_n^2);

% Do we have information?
if qfi==0
    disp('The Quantum Fisher information is zero.')
    return
end

% Quantum Cramer-Rao Bound
qcrb=1./(observations*qfi);

end
\end{lstlisting}

\subsection{Quantum Ziv-Zakai bound}
\label{subsec:qzzbnum}

\begin{lstlisting}[language=Matlab, mathescape=true]
function [qzzb] = mz_qzzb(initial_state,phase_width,mu_max)
% Quantum Ziv-Zakai bound as a function of the number of trials, where
% 'initial_state' is a pure state for the Mach-Zehnder interferometer,
% 'phase_width' is the width of the parameter domain and 'mu_max' is 
% the maximum number of repetitions. 

% Space cutoff (for a single mode)
op_cutoff=sqrt(length(initial_state));

% Parameter domain
W=phase_width;
dim_theta=1000;
theta=linspace(0,W,dim_theta);

% Fidelity
fidelity=zeros(dim_theta,1);
for z=1:dim_theta
    after_phase_shift=sparse(phase_shift_diff(op_cutoff,theta(z))$\hspace{0.15em}\swarrow$
*initial_state);
    fidelity(z)=abs(initial_state'*after_phase_shift)^2;
end

% Quantum Ziv-Zakai Bound integrand
integrand=zeros(dim_theta,mu_max);
for runs=1:mu_max
    for z=1:dim_theta
        integrand(z,runs)=0.5.*theta(z).*(1-theta(z)./W)$\hspace{0.15em}\swarrow$
.*(1-sqrt(1-fidelity(z).^runs));    
    end
end
integrand=sparse(integrand);

% Quantum Ziv-Zakai Bound
qzzb=trapz(theta,integrand,1);

end
\end{lstlisting}

\subsection{Quantum Weiss-Weinstein bound}
\label{subsec:qwwbnum}

\begin{lstlisting}[language=Matlab, mathescape=true]
function [qwwb] = mz_qwwb(initial_state,phase_width,mu_max)
% Quantum Weiss-Weinstein bound as a function of the number of 
% trials, where 'initial_state' is a pure state for the Mach-Zehnder,
% interferometer 'phase_width' is the width of the parameter domain 
% and 'mu_max' is the maximum number of repetitions. 

% Space cutoff (for a single mode)
op_cutoff=sqrt(length(initial_state));

% Parameter domain
W=phase_width;
dim_theta=1000;
h=linspace(0,W,dim_theta);

% Quantum Weiss-Weinstein Bound
find_supremum=zeros(mu_max,length(h));
qwwb=zeros(1,mu_max);
for runs=1:mu_max
    for z=1:length(h)
        zeta=initial_state'*phase_shift_diff(op_cutoff,h(z))*initial_state;
        zeta2=initial_state'*phase_shift_diff(op_cutoff,2*h(z))$\hspace{0.15em}\swarrow$
*initial_state;
        fid_function=abs(zeta)^2;
        find_supremum(runs,z)=h(z)^2*(1-h(z)/W)^2*fid_function^(2*runs)$\hspace{0.15em}\swarrow$
/(2*fid_function^runs-2*(1-2*h(z)/W)*real(zeta^(2*runs)*conj(zeta2)^runs));
    end
    qwwb(runs)=max(find_supremum(runs,:));
end

end
\end{lstlisting}

\section{Measurement strategies}
\label{sec:pomnum}

\begin{lstlisting}[language=Matlab, mathescape=true]
function [outcomes,proj_columns] = mz_pom$\hspace{0.15em}\swarrow$
(state_choice,pom_choice,phase_width,phase_mean)
% Outcomes and POM elements of five projective measurement schemes:
%
%   1) Optimal single-shot POM
%   2) 50:50 beam splitter + photon counting
%   3) 50:50 beam splitter + measurement of quadratures rotated by pi/8
%   4) Undoing the preparation of the initial state + photon counting
%   5) 50:50 beam splitter + parity measurements
%
% where 'state_choice' labels the initial state, 'pom_choice' selects one
% of the previous measurement schemes, 'phase_width' is the width of the 
% phase domain and 'phase_mean' is its centre. 
%
% Some extra phase shifts that are assumed to be known have been added 
% to 2) - 5) in order to make the strategy optimal when the prior is 
% centred around zero.

% Space cutoff (for a single mode)
op_cutoff=sqrt(length(initial_probe(state_choice)));

if pom_choice==1
    % 1) Optimal single-shot POM
    [~,outcomes,proj_columns,~,~,~,~,~,~]=mz_optimal_1trial(initial_probe$\hspace{0.15em}\swarrow$
(state_choice),phase_width,phase_mean);
    
elseif pom_choice==2
    % 2) 50:50 beam splitter + photon counting
        
    % Observable quantity (number of photons at each port)
    observable=kron(creation(op_cutoff)*creation(op_cutoff)',creation$\hspace{0.15em}\swarrow$
(op_cutoff)*creation(op_cutoff)');
    [proj_columns,outcomes_temp]=eig(full(observable));
    outcomes=zeros(1,length(outcomes_temp));
    for x=1:length(outcomes_temp)
        outcomes(x)=outcomes_temp(x,x);
    end
    
    % Extra phase shift
    odd_shift=kron(identity(op_cutoff),expm(1i*(pi/2)*creation$\hspace{0.15em}\swarrow$
(op_cutoff)*creation(op_cutoff)'));
    even_shift=kron(identity(op_cutoff),expm(1i*(pi/4)*creation$\hspace{0.15em}\swarrow$
(op_cutoff)*creation(op_cutoff)'));
    
    if state_choice==1
        optimal_shift=odd_shift;
    else
        optimal_shift=even_shift;
    end
    
    % Effect of the 50:50 beam splitter
    proj_columns=optimal_shift'*beam_splitter(op_cutoff)*proj_columns;
    
elseif pom_choice==3
    % 3) 50:50 beam splitter + measurement of quadratures rotated by pi/8
        
    % Observable quantity
    if state_choice==1
        error('The quadrature POM is not available for coherent states.')
    else
        phasequad1=pi/8;
        phasequad2=phasequad1;
    end
    quad1=(creation(op_cutoff)*exp(1i*phasequad1)+creation$\hspace{0.15em}\swarrow$
(op_cutoff)'*exp(-1i*phasequad1))/sqrt(2);
    quad2=(creation(op_cutoff)*exp(1i*phasequad2)+creation$\hspace{0.15em}\swarrow$
(op_cutoff)'*exp(-1i*phasequad2))/sqrt(2);
    observable=kron(quad1,quad2);
    [proj_columns,outcomes_temp]=eig(full(observable));
    outcomes=zeros(1,length(outcomes_temp));
    for x=1:length(outcomes_temp)
        outcomes(x)=outcomes_temp(x,x);
    end
    
    % Extra phase shift
    optimal_shift=kron(expm(-1i*(pi/4)*creation(op_cutoff)*creation$\hspace{0.15em}\swarrow$
(op_cutoff)'),identity(op_cutoff));
            
    % Effect of the 50:50 beam splitter
    proj_columns=optimal_shift'*beam_splitter(op_cutoff)*proj_columns;
    
elseif pom_choice==4
    % 4) Undoing the preparation of the initial state + photon counting
    if state_choice==1
    else
        error('This POM is only available for coherent states.')
    end  
    
    % Observable quantity (number of photons at each port)
    observable=kron(creation(op_cutoff)*creation(op_cutoff)',creation$\hspace{0.15em}\swarrow$
(op_cutoff)*creation(op_cutoff)');
    [proj_columns,outcomes_temp]=eig(full(observable));
    outcomes=zeros(1,length(outcomes_temp));
    for x=1:length(outcomes_temp)
        outcomes(x)=outcomes_temp(x,x);
    end
    
    % Extra phase shifts
    optimal_shift=sparse(expm(-1i*pi*j3schwinger(op_cutoff)));
    
    % Unitary transformations to undo the preparation of the state
    bs=sparse(beam_splitter(op_cutoff)');
    cs_undo=sparse(kron(displacement(op_cutoff,sqrt(2)),identity$\hspace{0.15em}\swarrow$
(op_cutoff)));
    combined=cs_undo*bs*optimal_shift;
    proj_columns=combined'*proj_columns; 
        
elseif pom_choice==5
    % 5) 50:50 beam splitter + parity measurements
    
    % Observable quantity (parity of the number of photons at each port)
    paritya=sparse(kron(identity(op_cutoff),(-1)^(full(creation$\hspace{0.15em}\swarrow$
(op_cutoff)*creation(op_cutoff)'))));
    parityb=sparse(kron((-1)^(full(creation(op_cutoff)*creation$\hspace{0.15em}\swarrow$
(op_cutoff)')),identity(op_cutoff)));
    observable=full(paritya*parityb);
    [proj_columns,outcomes_temp]=eig(full(observable));
    outcomes=zeros(1,length(outcomes_temp));
    for x=1:length(outcomes_temp)
        outcomes(x)=outcomes_temp(x,x);
    end
    
    % Extra phase shift
    odd_shift=kron(identity(op_cutoff),expm(1i*(pi/2)*creation$\hspace{0.15em}\swarrow$
(op_cutoff)*creation(op_cutoff)'));
    even_shift=kron(identity(op_cutoff),expm(1i*(pi/4)*creation$\hspace{0.15em}\swarrow$
(op_cutoff)*creation(op_cutoff)'));
    
    if state_choice==1
        optimal_shift=odd_shift;
    else
        optimal_shift=even_shift;
    end
    
    % Effect of the 50:50 beam splitter
    proj_columns=optimal_shift'*beam_splitter(op_cutoff)*proj_columns;  
end

end
\end{lstlisting}

\section{Prior information analysis}
\label{sec:priormatlab}

This algorithm generates the graphs in chapter \ref{chap:nonasymptotic} for the prior information analysis of the Mach-Zehnder interferometer. A version of this code was also employed for time estimation in section \ref{sec:fundtime}.

\begin{lstlisting}[language=Matlab, mathescape=true]
% Prior information analysis for single-parameter schemes
%
% This programme uses Bayes theorem to generate the posterior probability
%
% p(theta|m_1, ..., m_mu)
%
% for a flat prior and the likelihood function given by the Born rule. 
% The initial state and the measurement scheme are those of a Mach-Zehnder 
% interferometer, and they can be selected from the respective MATLAB 
% functions in our interferometric toolbox by giving a value from 1 to 5
% for 'state_choice' and 'pom_choice'.
%
% Important observations:
%
% - The prior is defined over all the parameter domain, so that the 
% symmetries of the likelihood that enable us to find the intrinsic 
% width can be visualised. 
%
% - The variables 'prior_mean_1shot' and 'prior_width_1shot' are needed
% to specify the optimal single-shot POM, but they do not affect the
% other measurement schemes. 
%
% - The results for the prior information analysis in chapter 4 are 
% recovered when we remove the extra phase shifts in the second option
% of our MATLAB function mz_pom(.) and we select it. This is because
% the results in chapter 4 were obtained for a prior between 0 and W, 
% while the POMs included in our sample of codes are those associated
% with a prior centred around zero (see chapter 5). 
clear

% State and POM options (see the respective codes in previous sections)
state_choice=2;
pom_choice=2;

% Initial state
initial_state=initial_probe(state_choice);

% Space cutoff
op_cutoff=sqrt(length(initial_state));

% Parameter domain
prior_mean=pi;
prior_width=2*pi; % Complete parameter domain
a=prior_mean-prior_width/2;
b=prior_mean+prior_width/2;
dim_theta=1000;
theta=linspace(a,b,dim_theta);

% Simulation of the unknown true value
index_real=160;

% Measurement scheme
prior_mean_1shot=0;
prior_width_1shot=pi/2; 
[outcomes_space,proj_columns] = mz_pom(state_choice,pom_choice,$\hspace{0.15em}\swarrow$
prior_width_1shot,prior_mean_1shot);

% State after the phase shift, final state and amplitudes
amplitudes=zeros(length(outcomes_space),dim_theta);
for z=1:dim_theta
    after_phase_shift=sparse(phase_shift_diff(op_cutoff,theta(z))$\hspace{0.15em}\swarrow$
*initial_state);
    for x=1:length(outcomes_space)
        pom_element=proj_columns(:,x);
        amplitudes(x,z)=sparse(pom_element'*after_phase_shift);
    end
end

% Likelihood function (using the Born rule)
likelihood=amplitudes.*conj(amplitudes);

% Prior density function
prior=ones(1,dim_theta);
prior=prior/trapz(theta,prior);

% Updating via Bayes theorem
prob_temp=prior;
for runs=1:100
    
    % Simulation of an interferometric experiment
    prob_sim=likelihood(:,index_real);
    cumulative = cumsum(prob_sim); % Cumulative function
    prob_rand=rand; % Random selection
    auxiliar=cumulative-prob_rand;
    
    for x=1:length(outcomes_space)
        if auxiliar(x)>0
            index=x;
            break
        end
    end
    
    % Posterior density function
    prob_temp=sparse(prob_temp.*likelihood(index,:));
    if trapz(theta,prob_temp)>1e-16
        prob_temp=prob_temp./trapz(theta,prob_temp);
    else
        prob_temp=0;
    end
    
    % Posterior probability plots
    if runs==1
        plot(theta,prob_temp,'k-','LineWidth',2.5)
        hold on
    elseif runs==2; plot(theta,prob_temp,'k-','LineWidth',2.5)
    elseif runs==10; plot(theta,prob_temp,'k-','LineWidth',2.5)
    elseif runs==100; plot(theta,prob_temp,'k-','LineWidth',2.5)
        hold off
    end
    
end

% Plot specifications
grid
fontsize=21;
set(gcf,'units','points','position',[250,50,550,400])
xlabel('$\color{mauve}{\$}$\theta$\color{mauve}{\$}$','Interpreter','latex','FontSize',fontsize)
ylabel('$\color{mauve}{\$}$p(\theta | \textbf{\textit{m}})$\color{mauve}{\$}$','Interpreter','latex',$\hspace{0.15em}\swarrow$
'FontSize',fontsize)
xticks([0 pi/2 pi 3*pi/2 2*pi])
xticklabels({'0','\pi/2','\pi', '3\pi/2','2\pi'})
xlim([min(theta) max(theta)])
set(gca, 'FontSize', fontsize,'FontName','Times New Roman')
\end{lstlisting}

\section{Mean square error for any number of trials}
\label{sec:msematlab}

\begin{lstlisting}[language=Matlab, mathescape=true]
function [epsilon_trials]=mz_mse_trials(state_choice,pom_choice,$\hspace{0.15em}\swarrow$
prior_width,prior_mean,mu_max)
% Bayesian mean square error
%
% This programme calculates the mean square error as a function of the
% number of repetitions.
%
% To run it, we need to specify the variables 'state_choice', which 
% labels the initial state of a Mach-Zehnder interferometer; 'pom_choice',
% which selects the measurement scheme; 'phase_width', which is the 
% width of a flat prior probability; 'phase_mean', which is the centre
% of its domain; and 'mu_max', which is the maximum number of trials.
%
% Note that this code relies on other MATLAB functions of our numerical
% toolbox. The algorithm in this section has been exploited to calculate
% the mean square error for all the single-parameter cases treated in this 
% thesis, including the ideal schemes for optical interferometry studied 
% in chapters 4 and 5, the calculation of the Taylor error to verify the
% validity of our squared approximation in appendix A, our lossy analysis 
% in chapter 8 and our analysis of the elapsed time, also in chapter 8.

% Seed for the random generator
rng('shuffle') 

% Initial state
initial_state=initial_probe(state_choice);

% Space cutoff
op_cutoff=sqrt(length(initial_state));

% Parameter domain
a=prior_mean-prior_width/2;
b=prior_mean+prior_width/2;
dim_theta=1250;
theta=linspace(a,b,dim_theta);
num_steps=125;
step=round(dim_theta/num_steps);
if step-round(step)~=0
    disp('Error: dim_theta divided by num_steps must be an integer.')
    return
elseif num_steps<3
    disp('Error: the approximation for the external theta integral needs$\hspace{0.15em}\swarrow$
three rectangles at least.')
    return
end

% Monte Carlo sample size
tau_mc=1250;    

% Measurement scheme
[outcomes_space,proj_columns] = mz_pom(state_choice,pom_choice,$\hspace{0.15em}\swarrow$
prior_width, prior_mean);

% State after the phase shift, final state and amplitudes
amplitudes=zeros(length(outcomes_space),dim_theta);
for z=1:dim_theta
    after_phase_shift=sparse(phase_shift_diff(op_cutoff,theta(z))$\hspace{0.15em}\swarrow$
*initial_state);
    for x=1:length(outcomes_space)        
        pom_element=proj_columns(:,x);
        amplitudes(x,z)=sparse(pom_element'*after_phase_shift);
    end             
end

% Likelihood function
likelihood=amplitudes.*conj(amplitudes);
disp('The likelihood function has been created.')
if (1-sum(likelihood(:,1)))>1e-7
    error('The quantum probabilities do not sum to one.')
end

% Prior probability
prior=ones(1,dim_theta);
prior=prior/trapz(theta,prior);

% Bayesian inference
epsilon_bar=0;
for index_real=1:step:dim_theta
    epsilon_n=zeros(1,mu_max); % Preallocate vector
    epsilon_n_sum=zeros(1,mu_max);
    for times=1:tau_mc
        
        % Prior density function
        prob_temp=prior;
        for runs=1:mu_max
            
            % (Monte Carlo) Interferometric simulation
            prob_sim1=likelihood(:,index_real);
            cumulative1 = cumsum(prob_sim1); % Cumulative function
            prob_rand1=rand; % Random selection
            auxiliar1=cumulative1-prob_rand1;
           
            for x=1:length(outcomes_space)
                if auxiliar1(x)>0
                    index1=x;
                    break
                end
            end
            
            % Posterior density function
            prob_temp=sparse(prob_temp.*likelihood(index1,:));
            if trapz(theta,prob_temp)>1e-16
                prob_temp=prob_temp./trapz(theta,prob_temp);
            else
                prob_temp=0;
            end
            
            % Experimental square error
            theta_expe=trapz(theta,prob_temp.*theta);
            theta2_expe=trapz(theta,prob_temp.*theta.^2);
            epsilon_n(runs)=theta2_expe-theta_expe^2;    
        end
        
        % Monte Carlo sum
        epsilon_n_sum=epsilon_n_sum+epsilon_n;
    end
       
    % Monte Carlo approximation for the Bayesian error
    epsilon_average=epsilon_n_sum/(tau_mc);
    epsilon_bar=epsilon_bar+epsilon_average*prior(index_real)$\hspace{0.15em}\swarrow$
*(theta(2*step)-theta(step));
end
epsilon_trials=epsilon_bar;
\end{lstlisting}

The numerical precision for our calculation of $\bar{\epsilon}_{\mathrm{mse}}(\mu)$ can be estimated using the identity
\begin{equation}
\int d\theta p(\theta) \theta^2 = \int d\theta' p(\theta') \int d\boldsymbol{m}~ p(\boldsymbol{m}|\theta') \int d\theta p(\theta|\boldsymbol{m})\theta^2,
\end{equation}
where the right hand side is to be calculated numerically (with a code analogous to that presented above) and to be compared to the analytical solution for the left hand side. We have found that our numerical results for the Mach-Zehnder interferometer are valid up to the third significant figure.  