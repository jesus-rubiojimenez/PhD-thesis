\chapter{Numerical toolbox for multi-parameter metrology}
\label{app:multinum}

\section{Multi-parameter prior information analysis}
\label{sec:multiprior}

\begin{lstlisting}[language=Matlab, mathescape=true]
% Two-parameter prior information analysis
%
% Prior information analysis for a qubit sensing network. The basic logic
% of the method parallels that for the single-parameter case (see appendix
% B.5 and chapter 6 for more details).

% Initial parameters
prior_mean1=pi; 
prior_mean2=prior_mean1;
prior_width1=2*pi;
prior_width2=2*pi;
mu_max=100;

% True values for the unknown parameters
theta1_real=1;
theta2_real=2;

% Initial state
gamma_par=1; % Local strategy
%gamma_par=0; % Maximally entangled strategy
%gamma_par=0.530696; % Asymptotically optimal strategy
%gamma_par=0.3343605926149827; % Balanced startegy
initial_state=sparse([1 gamma_par gamma_par 1])'/sqrt(2+2*gamma_par^2);

% Generators
sigmaz=sparse([1 0; 0 -1]);
g1=kron(sigmaz,identity(2))/2;
g2=kron(identity(2),sigmaz)/2;

% Asymptotically optimal local POM (F = F_q, chapter 6)
proj1=sparse([-1 -1 1 1])'/2;
proj2=sparse([1 1 1 1])'/2;
proj3=sparse([1 -1 -1 1])'/2;
proj4=sparse([-1 1 -1 1])'/2;
proj_columns=[proj1';proj2';proj3';proj4']';

% Optimal single-shot POM (chapter 7)
% proj1=sparse([1i 1 1 -1i])'/2;
% proj2=sparse([-1i 1 1 1i])'/2;
% proj3=sparse([1i -1 1 1i])'/2;
% proj4=sparse([-1i -1 1 -1i])'/2;
% proj_columns=[proj1';proj2';proj3';proj4']';

% Parameter domain
dim_theta=200;
a1=prior_mean1-prior_width1/2;
b1=prior_mean1+prior_width1/2;
theta1=linspace(a1,b1,dim_theta);
a2=prior_mean2-prior_width2/2;
b2=prior_mean2+prior_width2/2;
theta2=linspace(a2,b2,dim_theta);

% State after encoding the parameters, final state and amplitudes
amplitudes=zeros(size(proj_columns,2),dim_theta,dim_theta);
amplitudes_sparse=zeros(dim_theta,dim_theta,size(proj_columns,2));
for z1=1:dim_theta
    for z2=1:dim_theta
        after_encoding=sparse(expm(-1i*(g1*theta1(z1)+g2*theta1(z2))))$\hspace{0.15em}\swarrow$
*initial_state;
        for x=1:size(proj_columns,2)
            povm_element=proj_columns(:,x);     
            amplitudes_temp=sparse(povm_element)'*sparse(after_encoding);
            amplitudes(x,z1,z2)=amplitudes_temp;
            amplitudes_sparse(z1,z2,x)=amplitudes_temp;
        end

        % The second method of generating the amplitudes is included in 
        % order to use sparse later in the code.
    end
end

% Likelihood function
likelihood=amplitudes.*conj(amplitudes);
if (1-sum(likelihood(:,1,1)))>1e-7
    error('The quantum probabilities do not sum to one.')
end
likelihood_sparse=amplitudes_sparse.*conj(amplitudes_sparse);
if (1-sum(likelihood_sparse(1,1,:),3))>1e-7
    error('The quantum probabilities do not sum to one.')
end

% Prior probability
prior=ones(dim_theta,dim_theta);
prior=prior/trapz(theta2,trapz(theta1,prior));

% Simulation of the true values for the unkonwn parameters
for y=1:dim_theta 
    if theta1(y)>theta1_real || theta1(y)==theta1_real
        index_real1=y;
        break
    end
end

for y=1:dim_theta
    if theta2(y)>theta2_real || theta2(y)==theta2_real
        index_real2=y;
        break
    end
end

% Bayesian simulation
outcomes=zeros(1,mu_max);
for runs=1:mu_max
    
    % Simulation of the experimental outputs
    prob_sim=likelihood(:,index_real1,index_real2);
    cumulative1 = cumsum(prob_sim); % Cumulative function
    prob_rand=rand; % Random selection
    auxiliar=cumulative1-prob_rand;
    
    for x=1:size(proj_columns,2)
        if auxiliar(x)>0
            index=x;
            break
        end
    end
   
    outcomes(runs)=index;
end

% Prior density function
prob_temp=prior;
for runs=1:mu_max
    
    % Updated posterior density function
    ytemp=outcomes(runs);
    
    likesimulated=likelihood_sparse(:,:,ytemp);
    prob_temp=sparse(prob_temp.*likesimulated);
    prob_norm=sparse(trapz(theta2,trapz(theta1,prob_temp,1),2));
    if prob_norm>1e-16
        prob_temp=prob_temp/prob_norm;
    else
        prob_temp=0;
    end
    prob_temp=sparse(prob_temp);
end

% Plot of the posterior
contour(theta1',theta2',prob_temp,'LevelStep',0.1,'Fill','on')
xticks([0 pi/4 pi/2 3*pi/4 pi 5*pi/4 3*pi/2 7*pi/4 2*pi])
xticklabels({'0', '\pi/4', '\pi/2', '3\pi/4', '\pi', '5\pi/4', '3\pi/2',$\hspace{0.15em}\swarrow$
'7\pi/4','2\pi'})
yticks([0 pi/4 pi/2 3*pi/4 pi 5*pi/4 3*pi/2 7*pi/4 2*pi])
yticklabels({'0', '\pi/4', '\pi/2', '3\pi/4', '\pi', '5\pi/4', '3\pi/2',$\hspace{0.15em}\swarrow$
'7\pi/4','2\pi'})
xt = get(gca, 'XTick');
fontsize=32;
set(gca, 'FontSize', fontsize,'FontName','Times New Roman');
yt = get(gca, 'YTick');
set(gca, 'FontSize', fontsize,'FontName','Times New Roman');
grid
\end{lstlisting}

\section{Multi-parameter mean square error for any number of trials}
\label{sec:multimsematlab}

\begin{lstlisting}[language=Matlab, mathescape=true]
% Mean square error for the estimation of two linear functions
%
% The estimation scheme is a quantum sensing network with two qubits.
%
% Note that we use the trapezoidal rule 'trapz' for the inner parameter 
% integrals because these have peaked integrands, while Simpson's Rule 
% 'simps' is a better choice when this problem does not arise, which is
% the case for the outer parameter integrals.
clear

% Initial parameters
prior_mean1=pi/4;
prior_mean2=pi/4;
prior_width1=pi/2;
prior_width2=pi/2;
mu_max=1;

% Weighting matrix
WD=[1 0; 0 1]/2;

% Transformation representing the original parameters
%K=[1 0; 0 1];

% Transformation representing two linear functions
V=[2/sqrt(4+pi^2) 2/sqrt(5); pi/sqrt(4+pi^2) 1/sqrt(5)];

% Combination of linear transformation and weighting matrix
G=V*WD*V';

% Initial state
gamma_par=1; % Local strategy
%gamma_par=0; % Maximally entangled strategy
%gamma_par=0.530696; % Asymptotically optimal strategy
%gamma_par=0.3343605926149827; % Balanced startegy
initial_state=sparse([1 gamma_par gamma_par 1])'/sqrt(2+2*gamma_par^2);

% Generators
sigmaz=sparse([1 0; 0 -1]);
g1=kron(sigmaz,identity(2))/2;
g2=kron(identity(2),sigmaz)/2;

% Asymptotically optimal local POM (F = F_q, chapter 6)
proj1=sparse([-1 -1 1 1])'/2;
proj2=sparse([1 1 1 1])'/2;
proj3=sparse([1 -1 -1 1])'/2;
proj4=sparse([-1 1 -1 1])'/2;
proj_columns=[proj1';proj2';proj3';proj4']';

% Optimal single-shot POM (chapter 7)
% proj1=sparse([1i 1 1 -1i])'/2;
% proj2=sparse([-1i 1 1 1i])'/2;
% proj3=sparse([1i -1 1 1i])'/2;
% proj4=sparse([-1i -1 1 -1i])'/2;
% proj_columns=[proj1';proj2';proj3';proj4']';

% Parameter domain
dim_theta=100;
dim_theta_out=20;
a1=prior_mean1-prior_width1/2;
b1=prior_mean1+prior_width1/2;
theta1=linspace(a1,b1,dim_theta); % Inner parameter integrals
theta1_out=linspace(a1,b1,dim_theta_out); % Outer parameter integrals
a2=prior_mean2-prior_width2/2;
b2=prior_mean2+prior_width2/2;
theta2=linspace(a2,b2,dim_theta);
theta2_out=linspace(a2,b2,dim_theta_out);

% Monte Carlo sample size
tau_mc=200; 

% State after encoding the parameters, final state and amplitudes
amplitudes=zeros(size(proj_columns,2),dim_theta,dim_theta);
amplitudes_sparse=zeros(dim_theta,dim_theta,size(proj_columns,2));
for z1=1:dim_theta
    for z2=1:dim_theta
        after_encoding=sparse(expm(-1i*(g1*theta1(z1)+g2*theta1(z2))))$\hspace{0.15em}\swarrow$
*initial_state;
        for x=1:size(proj_columns,2)
            povm_element=proj_columns(:,x);     
            amplitudes_temp=sparse(povm_element)'*sparse(after_encoding);
            amplitudes(x,z1,z2)=amplitudes_temp;
            amplitudes_sparse(z1,z2,x)=amplitudes_temp;
        end

        % The second method of generating the amplitudes is included in 
        % order to use sparse later in the code.
    end
end

% Likelihood function
likelihood=amplitudes.*conj(amplitudes);
if (1-sum(likelihood(:,1,1)))>1e-7
    error('The quantum probabilities do not sum to one.')
end
likelihood_sparse=amplitudes_sparse.*conj(amplitudes_sparse);
if (1-sum(likelihood_sparse(1,1,:),3))>1e-7
    error('The quantum probabilities (sparse version) do not sum to one.')
end

% Prior probability
prior=ones(dim_theta,dim_theta);
prior=prior/trapz(theta2,trapz(theta1,prior));
prior_out=ones(dim_theta_out,dim_theta_out);
prior_out=prior_out/trapz(theta2_out,trapz(theta1_out,prior_out));

% Bayesian mean square error
epsilon_out=zeros(dim_theta_out,dim_theta_out);
for index_out1=1:dim_theta_out
    for index_out2=1:dim_theta_out
        
        % Matching outer and inner parameter indices       
        for y=1:dim_theta
            if theta1(y)>theta1_out(index_out1) ||$\hspace{0.15em}\swarrow$
theta1(y)==theta1_out(index_out1)
                index_real1=y;
                break
            end
        end
        
        for z=1:dim_theta
            if theta2(z)>theta2_out(index_out2) ||$\hspace{0.15em}\swarrow$
theta2(z)==theta2_out(index_out2)
                index_real2=z;
                break
            end
        end
        
        epsilon_n1=zeros(1,mu_max);
        epsilon_n2=zeros(1,mu_max);
        epsilon_n_offdia=zeros(1,mu_max);
        epsilon_n_sum=zeros(1,mu_max);
        for times=1:tau_mc
            
            % Prior density function
            prob_temp=sparse(prior);
            for runs=1:mu_max
                
                % (Monte Carlo) Outcome simulation
                prob_sim=likelihood(:,index_real1,index_real2);
                cumulative = cumsum(prob_sim); % Cumulative function
                prob_rand=rand; % Random selection
                auxiliar=cumulative-prob_rand;
                
                for x=1:size(proj_columns,2)
                    if auxiliar(x)>0
                        index_mc=x;
                        break
                    end
                end
                
                % Posterior density function
                likesimulated=likelihood_sparse(:,:,index_mc);
                prob_temp=sparse(prob_temp.*likesimulated);
                normalisation=sparse(trapz(theta2,trapz$\hspace{0.15em}\swarrow$
(theta1,prob_temp,1),2));
                if normalisation>1e-16
                    prob_temp=prob_temp/normalisation;
                else
                    prob_temp=0;
                end
                prob_temp=sparse(prob_temp);
                
                % Bayes estimator for the first parameter
                theta_expe1=trapz(theta1,trapz(theta2,prob_temp,2)$\hspace{0.15em}\swarrow$
.*theta1',1);
                theta2_expe1=trapz(theta1,trapz(theta2,prob_temp,2)$\hspace{0.15em}\swarrow$
.*theta1'.^2,1);
                epsilon_n1(runs)=theta2_expe1-theta_expe1^2;
                
                % Bayes estimator for the second parameter
                theta_expe2=trapz(theta2,trapz(theta1,prob_temp,1)$\hspace{0.15em}\swarrow$
.*theta2,2);
                theta2_expe2=trapz(theta2,trapz(theta1,prob_temp,1)$\hspace{0.15em}\swarrow$
.*theta2.^2,2);
                epsilon_n2(runs)=theta2_expe2-theta_expe2^2;
                
                % Off-diagonal terms (the covariance matrix is symmetric)
                theta2_offdia=trapz(theta1,trapz(theta2,prob_temp$\hspace{0.15em}\swarrow$
.*theta2,2).*theta1',1);
                epsilon_n_offdia(runs)=theta2_offdia-theta_expe1$\hspace{0.15em}\swarrow$
*theta_expe2;

            end
            
            % Monte Carlo sum with transformation and weighting matrices
            epsilon_n_sum=epsilon_n_sum+G(1,1)*epsilon_n1+G(2,2)$\hspace{0.15em}\swarrow$
*epsilon_n2+2*G(1,2)*epsilon_n_offdia;
        end
                  
        % Monte Carlo approximation
        epsilon_average=epsilon_n_sum/(tau_mc);
        for runs_out=1:mu_max
            epsilon_out(index_out1,index_out2,runs_out)$\hspace{0.15em}\swarrow$
=epsilon_average(runs_out);
        end
    end
end

% Outer integral
epsilon_trials=zeros(1,mu_max);
for runs_out=1:mu_max
    epsilon_temp=epsilon_out(:,:,runs_out);
    epsilon_trials(runs_out)=simps(theta2_out,simps(theta1_out,prior_out$\hspace{0.15em}\swarrow$
.*epsilon_temp));
end

% Observations
observations=1:1:mu_max;

% Fisher information matrix
F11=4*(initial_state'*g1^2*initial_state-(initial_state'$\hspace{0.15em}\swarrow$
*g1*initial_state)^2);
F12=4*(initial_state'*g1*g2*initial_state-(initial_state'$\hspace{0.15em}\swarrow$
*g1*initial_state)*(initial_state'*g2*initial_state));
F21=4*(initial_state'*g2*g1*initial_state-(initial_state'$\hspace{0.15em}\swarrow$
*g2*initial_state)*(initial_state'*g1*initial_state));
F22=4*(initial_state'*g2^2*initial_state-(initial_state'$\hspace{0.15em}\swarrow$
*g2*initial_state)^2);
F=[F11 F12; F21 F22];

% Quantum Cramer-Rao bound
qcrb=trace(G/F)./(observations);

% Save results
%save('qnetwork_results.txt','observations','epsilon_trials','qcrb',$\hspace{0.15em}\swarrow$
'-ascii')
\end{lstlisting}