\chapter{Conclusions}
\label{chap:conclusions}

Every journey has a final destination, and it is time for us to reach ours. Our particular journey started with the realisation that, ultimately, the success of science as a method to understand the world stems from the solid foundation provided by the empirical facts that we are able to identify. Our ability to extract information from reality is thus crucial, and this is precisely where quantum metrology enters the scene as one of our best frameworks to enhance our ways of communicating with nature, which is achieved via measurements that in this case rely on the quantum properties of matter and light. 

Unfortunately, nature does not always provide us with as much information as we would like to have, and this has two fundamental consequences. On the  one hand, it means that the amount of empirical data that a certain experiment is allowed to extract might be very limited. On the other hand, the available prior knowledge will in many cases be moderate at best, since the prior information is essentially a manifestation of what we learned from either previous experiments or theories whose validity is grounded on empirical evidence.

Given that many quantum protocols are currently devised assuming either an abundance of measurement data, or a very good prior information, or perhaps both, it was crucial to revisit the techniques of quantum metrology and extend them such that they could be efficiently and reliably applied to scenarios where the previous limitations are present. This is exactly what the research in this thesis has achieved.  

The path that we have followed to solve this problem has led us to a new methodology that has opened the door to an alternative way of doing quantum metrology, and that promises to play a central role in any future study of non-asymptotic protocols designed for scenarios with a limited amount of data. That this is likely to be the case has in fact been demonstrated through many surprising new results that have emerged from the application of our methods to specific metrology schemes. 

One of the systems that we have studied in more depth is the Mach-Zehnder interferometer, which is a paradigmatic scheme in the context of optical interferometry. Here the goal is to provide a good estimate for a parameter that represents the difference of optical phase shifts between the arms of the interferometer. Focusing our attention on state-of-the-art probe states that can be constructed with operations such as displacements of the vacuum or squeezing, we have first shown that the number of repetitions and the minimum amount of prior knowledge that are needed for the asymptotic theory to be meaningful crucially depend on the specific properties that a given probe has. For example, given a fixed amount of prior knowledge, we have found that while common probes such as coherent states might require a small number of  trials to reach the performance predicted by the asymptotic theory, more exotic cases such as the squeezed entangled state present a much slower convergence. The crucial observation is that the latter promises a great precision-enhancement with respect to coherent states when we only look at the asymptotic theory, and yet the coherent state beats the squeezed entangled state when the data is limited and we perform a standard photon-counting measurement. The general conclusion is that the ordering of states in terms of their performance is dramatically affected by the number of times that the experiment is repeated, and thus maximising the Fisher information might not always be the best approach. 

The previous idea has been put on a more solid basis by means of our shot-by-shot optimisation method, that is, by repeating the quantum strategy that is optimal for a single shot. Remarkably, while this is a fully Bayesian approach that in principle does not rely on the Fisher information, our method sometimes recovers the predictions of the latter, either in the limit of a large number of repetitions, or in the limit of a narrow prior for a single shot. A crucial finding derived from this approach is the evidence for the existence of a trade-off between the asymptotic and non-asymptotic uncertainties. In particular, we have found that increasing the amount of photon correlations within each of the modes of the interferometer might be detrimental if the experiment is operating with a limited amount of data, despite the fact that these correlations are known to be extremely useful once we have reached the asymptotic regime. More surprisingly, the calculation of a state with less intra-mode correlations but a certain amount of mode entanglement has proven to be a better choice to keep the precision high in both the asymptotic and non-asymptotic regimes, even when the asymptotic theory indicates that mode entanglement only provides a limited advantage. As a consequence, our results indicate that we should pay more attention to the amount of data as a feature that might alter our assessment of the role of correlations in quantum metrology. 

On the other hand, we have shown that our shot-by-shot approach is a useful tool to generate precision bounds that, at least for repetitive experiments, have a certain fundamental character, and we have demonstrated that our bounds can be tighter than other proposals in the literature such as the quantum versions of the Ziv-Zakai and Weiss-Weinstein bounds. Interestingly, our bound for the NOON state is also its true fundamental limit, since we have shown that, in this case, general collective measurements are not better than simply repeating the single-shot optimal strategy. On a more practical note, our bounds have proven to be very useful to assess how close to the precision limits that we have calculated the uncertainty associated with practical measurements can be. For example, we have found that while measuring quadratures and a measurement based on counting photons are, for ideal schemes, equally precise when the scheme operates in the asymptotic regime, the former is closer to our bounds when the number of repetitions is small. In addition, by combining our method with a genetic algorithm we have provided sequences of operations to generate probes that not only have a good performance in the regime of limited data, but that may also be implemented in the laboratory with current technology. In other words, our non-asymptotic analysis of the Mach-Zehnder interferometer has revealed new theoretical properties about the interplay between amount of data, prior information and photon correlations that were previously unknown, and it has also provided specific procedures that may be relevant in real-world implementations of our protocols when these operate in the non-asymptotic regime.  

Given our aim of bringing quantum metrology techniques closer to the reality of experimental practice, our methodology would not have been complete if we had not addressed multi-parameter metrology problems, since many practical applications require the estimation of several pieces of information. In this context, we have chosen to focus on the design of quantum sensing networks, which is a model for distributed sensing. The implementation of this type of configuration might involve large distances between the quantum sensors that form the network (this is the case, e.g., in a network of satellites), and this makes the construction  and maintenance of these quantum networks potentially challenging. For that reason, it was crucial to identify strategies that can perform optimally even when the available resources are limited, including both the number of times that the protocol can be run and the amount of correlations between the sensors that we may have. The presence of several parameters provides, in addition, a set of possibilities to enhance the protocol that is larger than in the single-parameter case, and it is useful to split the problem in two parts. The first of them involves the estimation of properties that have been locally encoded in each sensor. Applying our shot-by-shot method we have shown that, in that case, entanglement between sensors is not required to achieve the optimal precision that a network of qubits could provide. In addition, we have found that neither is entanglement necessary to benefit from the precision-enhancement associated with a quantum imagining protocol when the latter is compared to the individual estimation based on Mach-Zehnder interferometers. Remarkably, this is the same conclusion that had been reached previously in the literature in the context of the asymptotic theory. Therefore, our result has effectively extended such conclusion to the regime of limited data and a moderate amount of prior knowledge.

The second part of the problem of quantum sensing networks involves the estimation of properties that are modelled by arbitrary functions of several locally-encoded parameters. For that reason, we may say that such functions represent global properties of the network. It was known that entanglement sometimes enhances the performance of the schemes designed for this specific problem notably, but the situations where this had been shown were mostly limited to considering a single function. Here we have been able to go a step further. In particular, we have considered linear but otherwise arbitrary functions, and we have solved the asymptotic estimation problem completely for the particular case of sensor-symmetric networks, which can be seen as a generalisation of the symmetric configurations that are typically utilised in optical interferometry. These asymptotic solutions were then employed as a guide to perform our non-asymptotic analysis, and we have shown that the amount of inter-sensor correlations that is optimal crucially depends on the number of repetitions and the prior information that it is being assumed, which is exactly the same type of phenomenon that we had uncovered for the Mach-Zehnder interferometer. For example, we have found that if the vectors formed with the components of functions are clustered around the direction associated with maximally entangled states, then these probes will be the best choice for a small number of trials and a vague prior, while only a moderate amount of positive correlations will be required to achieve the asymptotic optimum. Taking into account that the same type of behaviour has been established both for single-parameter and multi-parameter schemes, which were, in addition, based on physically different systems, our results suggest that the interplay between correlations, amount of data and prior knowledge is in fact a more general feature, so that we may expect it to also arise in other estimation problems. It appears that if we could learn how to control the aforementioned interplay in practice, then we would have at our disposition a remarkably large amount of unexplored possibilities to enhance non-asymptotic quantum protocols.

If we look at our results from a more fundamental point of view, then we can see that two very satisfactory features of the methodology that we have developed are its unified character and consistency. Indeed, the path that we have followed has effectively transformed an initial collection of techniques - many of them already known but often treated as if they were unrelated to each other - into a unified framework that offers a much broader perspective. We have seen that, for us, the first question to be asked before we start the optimisation of our protocols is what is the physically meaningful quantity that we should employ to assess the uncertainty, and whether we choose to rely on bounds or on any other technique is mostly related to which tools generate more tractable calculations. Our formalism then follows naturally from this point of view: given a measure of uncertainty that we wish to use with a class of protocols based on repetitive experiments, we can either optimise the system in a shot-by-shot fashion, which is arguably the most general and fundamental possibility for our particular case, or we can follow a weaker approach and only require that the protocol performs optimally as the data accumulate. We have shown that these simple but powerful ideas can be applied to both single-parameter and multi-parameter cases, and we have even derived a new multi-parameter quantum bound during the process of adapting our methods to the latter case. We can conclude that, as we announced in the introduction, we have proposed, constructed, explored and exploited a \emph{non-asymptotic quantum metrology}.